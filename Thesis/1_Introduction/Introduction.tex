When the bid to co-host the Square Kilometre Array (SKA) in Southern Africa was won in 2012, the SKA Africa team celebrated a great achievements. Seven years of hard work had rewarded them with the opportunity to build the largest scientific instrument in human history. But with it came the challenges of achieving the goals set out by the SKA; engineering, data capture and data processing on a scale which had never before been thought of, let alone attempted, would be needed to see these goals come to fruition.
\\
\\
Galaxy formation and evolution, life elsewhere in the universe and a deeper understanding of dark matter and dark energy\footnotemark are just some of areas the SKA could shed light on. But answering questions like these can be difficult unless the data obtained from the array of antennae making up the SKA is clean and accurate.
\footnotetext{Taken from \url{http://www.ska.ac.za/about/faqs/}}
\\
\\
Part of the solution to providing an accurate data set operate on, arises from the need to correct \gls{dd} effects \citep{smirnov2011revisiting}. \gls{dd} effects stem from interference radio waves experience while passing through the upper layers of the atmosphere and imperfections in the rotating mounts the antennae are mounted on. These distortions manifest themselves as blurring in images generated by the data.
\\
\\
While means to correct these errors exist, they can be slow, work independently to the data they operate on or both. The best current model uses Voronoi tessellations \citep{okabe2009spatial} to generate a set of facets, which do not overlap, that are then independently processed for cleaning to produce an overall improved image. While this method works well, more can be done to ensure the dimensions of the facets are optimal to ensure the cleaning is more effective and the data as accurate as possible.
%%%%%%%%%%%%%%%%%%%%%%%%%%%%%%%%%%%%%%%%%%%%%%%%%%%%%%%%%%%%%%%%%%%%%%%%%
\section{Research Objectives}
The research aims to investigate and achieve the following goals:
\begin{enumerate}
\item The Voronoi tessellation algorithm will be improved on to achieve an easily scalable algorithm which completes tessellation in a faster computational time while also being parallisable for future improvements.
\item Implementing a means to abstractly measure the effectiveness of a given tessellation structure on a set of sources with a fixed number of facets.
\item An effort will be made to improve the faceting process overall to increase the effect had by individual data sources within the space on the final structure of the tessellation without increasing the total number of facets.
\item Using data parallelism on a \gls{gpu} to further increase the performance and decrease the computation time of the faceting process at key points.
\end{enumerate}
%%%%%%%%%%%%%%%%%%%%%%%%%%%%%%%%%%%%%%%%%%%%%%%%%%%%%%%%%%%%%%%%%%%%%%%%%
\section{Proposed Approach}
In order to better understand the problem, research into the SKA, radio astronomy in general and specifically the image capturing process is needed first. In order to create an improved faceting algorithm that tessellates the plane efficiently and effectively, an understanding of the different available faceting algorithms is required and will be researched. A metric will be required to analyse the effectiveness of the tessellation, this metric must be sensitive to the unique structure of the data sources, the means in which the data is being captured and the distortion which is being corrected for. Once the tessellation algorithm and metric are obtained, a means of clustering the data in an effective manner, in order to improve the tessellation, is required. As stated in the objective, this clustering method must allow the overall structure of the tessellation to be proportionally affected by each data source, no matter how minor. Once these are complete, a means to improve the performance of the algorithm will be looked into by studying the NVIDIA GPU architecture and the CUDA GPU programming language as well as other data parallelism paradigms.
%%%%%%%%%%%%%%%%%%%%%%%%%%%%%%%%%%%%%%%%%%%%%%%%%%%%%%%%%%%%%%%%%%%%%%%%%
\section{Document Structure}
The remainder of this document is designed to address the issues stated above and lead the reader through the investigation process to achieve the objectives set out. It is structured as follows:
\\
\\
Chapter 2 includes literature used to build knowledge on the means to solve the problems at hand. It is divided into three main sections, and begins by discussing radio astronomy, the structure of the telescope, the process of capturing and processing images and naive methods for correcting \gls{dd} effects. It then proceeds to discuss Voronoi tessellations, extensions and algorithms for generating Voronoi tessellations and several clustering algorithms. The chapter concludes by discussing data parallelism, NVIDIA GPUs, CUDA and several optimizations for GPUs.
\\
\\
Chapter 3 discusses the design of the proposed solution to the problem including some pitfalls and problems in the process. It begins by discussing the means by which data sources are selected to generate the tessellation. It then discusses the algorithm of the improved Voronoi tessellation, it's structure, creation and problems. The metric for computing the effectiveness of the algorithm is then evaluated and discussed as well as how it can be used to improve the tessellation overall. It then goes on to discuss the facet merging operation, which is used to improve the overall structure of the faceting. It discusses how the operation finds and executes the optimal merge for the tessellated structure. Lastly, it discusses how the GPU was used to optimise key points of the algorithm through concurrent data processing. It discusses the libraries that were used as well as how the algorithm was converted from executing only on a CPU to being executed on both a CPU and a GPU.
\\
\\
Chapter 4 discusses the results obtained from comparisons between the naive tessellation, the simple Voronoi tessellation and the facet merging tessellation created in Chapter 3. It compares their performance for a given number of facets using the performance metric created in Chapter 3. It also compares the efficiency and speed of the GPU implementation to that of the standard CPU implementation.
\\
\\
Chapter 5 concludes the paper by analysing and discussing the results obtained in Chapter 4. It discusses problems with the solution obtained as well as its successes. It concludes by noting future improvements and approaches to solve the problems which have arisen and further improve the results.
%%%%%%%%%%%%%%%%%%%%%%%%%%%%%%%%%%%%%%%%%%%%%%%%%%%%%%%%%%%%%%%%%%%%%%%%%
