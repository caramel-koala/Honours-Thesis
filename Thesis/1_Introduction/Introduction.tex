When the bid to co-host the Square Kilometre Array (SKA) in Southern Africa was won in 2012, the SKA Africa team celebrated a great achievement. Seven years of hard work had rewarded them with the opportunity to build the largest scientific instrument in human history. But with this came the challenges of achieving the goals set out by the SKA; engineering, data capture and data processing on a scale which had never before been thought of, let alone attempted, would be needed to see these goals come to fruition.
\\
\\
Galaxy formation and evolution, life elsewhere in the universe and a deeper understanding of dark matter and dark energy\footnotemark are just some of the areas the SKA could shed light on. But answering questions like these can be difficult unless the data obtained from the array of antennae making up the SKA is clean and accurate.
\footnotetext{Taken from \url{http://www.ska.ac.za/about/faqs/}}
\\
\\
Part of the solution to providing an accurate data set to operate on, arises from the need to correct \gls{dd} effects \citep{smirnov2011revisiting}. \gls{dd} effects stem from interference radio waves experience while passing through the upper layers of the atmosphere and imperfections in the rotating mounts the antennae are mounted on. These distortions manifest themselves as blurring in images generated by the data.
\\
\\
While means to correct these errors exist, they can be slow, work independently of the data they operate on or both. The best current model uses a Voronoi tessellation \citep{okabe2009spatial} to break the image into processable units, which do not overlap, and which are independently processed for cleaning to produce an overall improved image. While this method works well, more can be done to ensure the is broken up optimally which, in turn, makes the cleaning more effective and the data as accurate as possible.
%%%%%%%%%%%%%%%%%%%%%%%%%%%%%%%%%%%%%%%%%%%%%%%%%%%%%%%%%%%%%%%%%%%%%%%%%
\section{Research Statement}
This thesis will seek to improve the current model for breaking up the image. It will do so by exploring geometric concepts to improve the structure of the pieces and parallel computation to improve the execution time of the process.
\section{Research Objectives}\label{int:sec:goals}
Given the research statement above, the objectives of this research are the following:
\begin{enumerate}
\item Improving the Voronoi tessellation algorithm to achieve an easily scalable algorithm which completes tessellation in a faster computational time while also being parallelisable for future improvements.
\item Implementing a means to abstractly measure the effectiveness of a given tessellation structure on a set of sources with a fixed number of facets.
\item Overall improvement to the faceting process to increase the effect that individual data sources within the space have on the final structure of the tessellation without increasing the total number of facets.
\item Using data parallelism on a \gls{gpu} to further increase the performance and decrease the computation time of the tessellation process at key points.
\end{enumerate}
%%%%%%%%%%%%%%%%%%%%%%%%%%%%%%%%%%%%%%%%%%%%%%%%%%%%%%%%%%%%%%%%%%%%%%%%%
\section{Proposed Approach}
To better understand the problem, research on radio astronomy in general, and specifically the image capturing process was first needed. In order to create an improved tessellation algorithm that tessellates the plane efficiently and effectively, an understanding of the different available tessellation algorithms was required. A metric was used to analyse the effectiveness of the tessellation; this metric needed to be sensitive to the unique structure of the data sources, the means through which the data are captured and the distortion which is being corrected for. Once the tessellation algorithm and metric have been obtained, a means of clustering the data in an effective manner, in order to improve the tessellation, was sought. As stated in the objectives, this clustering method should allow the overall structure of the tessellation to be proportionally affected by each data source. Once these steps have been completed, a means to improve the performance of the algorithm was investigated by studying the NVIDIA GPU architecture and the CUDA GPU programming language as well as other data parallelism paradigms.
%%%%%%%%%%%%%%%%%%%%%%%%%%%%%%%%%%%%%%%%%%%%%%%%%%%%%%%%%%%%%%%%%%%%%%%%%
\section{Structure of Thesis}
The remainder of this thesis is designed to address the issues stated above and lead the reader through the investigation process to achieve the stated objectives. It is structured as follows:
\\
\\
Chapter 2 includes literature used to build knowledge on the means to solve the problems at hand. It is divided into three main sections, and begins by discussing radio astronomy, the structure of the telescope, the process of capturing and processing images and naive methods for correcting \gls{dd} effects. It proceeds to discuss Voronoi tessellations, extensions and algorithms for generating Voronoi tessellations and several clustering algorithms. The chapter concludes by discussing data parallelism, NVIDIA GPUs, CUDA and several optimizations for GPUs.
\\
\\
Chapter 3 discusses the design of the proposed solution to the problem including some pitfalls and problems in the process. It begins by discussing the means by which data sources are selected to generate the tessellation. It discusses the algorithm of the improved Voronoi tessellation, its structure, creation and problems. The metric for computing the effectiveness of the algorithm is evaluated and discussed as well as how it can be used to improve the tessellation overall. The chapter goes on to discuss the merging operation, which is used to improve the overall structure of the tessellation. It discusses how the operation finds and executes the optimal merge for the tessellated structure.
\\
\\
Chapter 4 discusses how the GPU was used to optimise key points of the algorithm through concurrent data processing. It discusses the libraries that were used as well as how the algorithm was converted from executing only on a CPU to being executed on both a CPU and a GPU.
\\
\\
Chapter 5 discusses the results obtained from comparisons between the simple Voronoi tessellation and the cell merging tessellation created in Chapters 3 and 4. It compares their performance for a given number of cells using the performance metric created in Chapter 3. It also compares the efficiency and speed of the GPU implementation to that of the standard CPU implementation.
\\
\\
Chapter 6 concludes the thesis by analysing and discussing the results obtained in Chapter 5. It discusses problems with the solution obtained as well as its successes. It concludes by noting future improvements and approaches to solve the problems that have arisen and further improve the results.
%%%%%%%%%%%%%%%%%%%%%%%%%%%%%%%%%%%%%%%%%%%%%%%%%%%%%%%%%%%%%%%%%%%%%%%%%
