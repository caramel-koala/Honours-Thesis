It is evident that areas of the process can be optimised by using a parallel processing unit such as a GPU. Areas that can be optimised include the generation of the Voronoi, the divide and conquer can spawn two threads to compute the left and right diagrams, it would then simply take the diagrams returned by the threads it spawned, perform the merge operation and pass it back to it's parent thread. CUDA is limited to a nested depth of 24 calls \citep{CUDA}, which therefore allows for up to $2^{25}$ or $33554432$ centres, this exceeds the expected number of centres since the number of sources is generally $~10000$ which is far below this. It should be noted that the GPU, with $2$GB of memory, will likely run out of memory before the maximum recursive depth is reached. The maximum number of centres is set to $2^{25}$ instead of $2^{24}$ since the minimum number of points needed to generate a Voronoi tessellation for a given thread at the base case is two.
\\
\\
The other possible case for GPU parallelisation is determining the best possible merge to be executed. The merge operation itself cannot occur in parallel this may lead to conflicting merges if a single cell can optimally merge with multiple neighbours. For the case of the merge, since there is very little nested function calling, the only restriction on our data is the memory capacity of the GPU. For this research, we will only look at parallel computation of finding the best merge.