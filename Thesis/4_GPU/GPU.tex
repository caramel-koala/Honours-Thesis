It is possible that areas of the process could be optimised by using a parallel processing unit such as a GPU. A potential for optimisation is the generation of the Voronoi. The divide-and-conquer algorithm can spawn two threads to compute the left and right diagrams, recursively. The diagrams returned by the threads would be merged and pass it back to the parent thread. CUDA is limited to a nested depth of 24 calls \citep{CUDA}, allowing for up to $3 \times 2^{24}$ or $50331648$ centres, which exceeds the expected number of centres that is $\sim10000$. It should however be noted that the GPU, with $2$ GB of memory, will likely run out of memory before the maximum recursive depth is reached. The maximum number of centres is set to $3 \times 2^{24}$ as the maximum recursive depth is 24, allowing for $2^24$ branches in the divide-and-conquer and the base case of the divide and conquer can handle up to three centres.
\\
\\
The second possible area for GPU parallelisation is determining the best possible merge to be executed. The merge operation itself cannot occur in parallel as this could lead to conflicting merges if a single cell can optimally merge with multiple neighbours. For the merge algorithm, since there are very few nested function calling, the only restriction on our data is the memory capacity of the GPU. For this research, we will only look at parallel computation of finding the best merge candidates.