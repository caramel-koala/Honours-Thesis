\section{Summary}\label{sum}
In this chapter we have discussed many of the technical and theoretical aspects used in this study. We began by looking at radio astronomy and how radio telescopes are designed as parabolic arcs. We reviewed Voronoi diagrams and discussed how power diagrams are the natural extension thereof when weights are incorporated. We looked at three clustering algorithms, namely the k-means, the agglomerative, and the bisecting k-means algorithms. Lastly we looked at GPUs, their architecture, and some of the concepts involved. We discussed parallelism and how it has become a necessity due to the physical limitations of individual processors.
\\
\\
At each stage we also looked at existing models, similar to what has been proposed. We looked at the naive grid method for DD-effect error correction and its shortcomings due to empty blocks and off-center optimal points. We discussed a standard Voronoi model for use in the correction of DD effects, saw how it improved on the naive method, but also how it fell short by only regarding the brightest points which may lie too close to one another and warp the entire polygon. Finally, we looked at the jump flood algorithm as a GPU based Voronoi tessellation algorithm. This algorithm approximates a tessellation by changing the space to a grid of finite cells and uses decremental steps to generate a tessellation in a reasonable time.
%--------------------------------------------------------------------------------------