\section{Summary}\label{sum}
In this chapter we have discussed many of the technical and theoretical aspects of the algorithm which needs to be developed. We began by looking at radio astronomy and how radio telescopes are designed as parabolic arcs. We reviewed Voronoi diagrams and discussed how power diagrams are their natural extension when weights are incorporated. We looked at three algorithms again, namely the k-means, the agglomerative, and the bisecting k-means algorithms. Lastly we looked at GPUs, their architecture, and some of the concepts involved. We discussed parallelism and how it has arisen as a computational norm from the issue of frequency CPUs are unable to overcome.
\\
\\
At each stage we also looked at existing models which do work similar to what will be done in the later paper. We looked at the naive grid method for DD-effect error correction and its shortcomings due to empty blocks and off center optimal points. We discussed a standard Voronoi model for use in the correction of DD effects, we saw how it improved on the naive method but also how it fell short with it only regarding the brightest points which may lie too close to one another and warp the entire polygon. We lastly looked at the jump flood algorithm as a GPU based voronoi tessellation algorithm, we saw that it approximated a tessellation by changing the space to a grid of finite cells and used decremental steps to generate a tessellation in a reasonable time.
%--------------------------------------------------------------------------------------