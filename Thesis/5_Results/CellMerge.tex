\section{Cell Merge Performance}
In order to compare the performance of the cell merge to that of the Voronoi tessellation a set of 1000 unique sources was randomly generated. Using these sources, 936 tessellations were generated using each algorithm giving 64 to 1000 cells per generation. For the Voronoi, the brightest sources were chosen as centres for tessellation for each iteration. The Voronois were re-centred to minimise the error as much as possible. For each iteration of the cell merge, the space was tessellated using all 1000 sources as centres and the merge process was iterated until the desired number of cells was reached. For each iteration of both algorithms, the total error for the space was calculated and stored. While both algorithms place their centres optimally within their cells, the structure of the cells generated in the cell merge are dependent on the intensities and positions of all sources in the space while those in the Voronoi are generated in a way that is dependent on only the positions of the sources with the highest intensities. Figure \ref{res:fig:error} shows the error for a set number of cells for a fixed set of sources and Figure \ref{res:fig:error_log} shows the logarithmic scaled version of the resulting error data.
\begin{figure}[H]
\centering
\includegraphics[width=0.8\textwidth]{Images/result_error.png}
\caption{A comparison of the total error of the Voronoi tessellation (red) and the cell merge (blue) algorithms.}
\label{res:fig:error}
\end{figure}
\begin{figure}[H]
\centering
\includegraphics[width=0.7\textwidth]{Images/result_error_log.png}
\caption{A logarithmic scale of the data presented in Figure \ref{res:fig:error}.}
\label{res:fig:error_log}
\end{figure}
As shown, both algorithms have zero error when generating 1000 cells but once the number of cells decrease, the error begins increasing exponentially. Figure \ref{res:fig:error} therefore shows that the cell merge has a lower error compared to that of the Voronoi for a high number of centres. Figure \ref{res:fig:error_log_z} shows that for a very low number of cell, the Voronoi becomes more feasible. While both still have extremely large errors, the Voronoi's is lower than that of the cell merge until they once again meet when a single cell is used to generate the Voronoi.
\begin{figure}[H]
\centering
\includegraphics[width=0.8\textwidth]{Images/result_error_log_zoom.png}
\caption{A close up of the logarithmic scale data showing the point at which the cell merge and Voronoi intercept.}
\label{res:fig:error_log_z}
\end{figure}