\section{Source Selection}
We begin the algorithm with a list $n$ sources in the plane. Each sources has 3 parameters: $x$, $y$ and $z$. The $x$ and $y$ parameters define the spatial coordinates of the source while the z parameter is seen as the intensity of the source.
\\
\\
The sources are read in and are used to generate the Voronoi centres. Each centre has the following attributes:
\begin{enumerate}
\item An $x$, $y$, and $z$ value inherited from its source.
\item A boolean to determine if it is the circumcenter which defaults to false.
\item A centre clockwise to it if it lies on the convex hull which is defaulted to be empty.
\item A centre counter-clockwise to it if it lies on the convex hull which is defaulted to be empty.
\item A list of all its neighbouring points and the line that bisects them which is also set to empty by default.
\item A list of sources in the cell created by the centre which is set to empty by default.
\item The cell's error which is set to zero by default as there are no sources yet.
\item A list of all the cells which have been merged with this cell, by default, this is also set to be empty.
\end{enumerate} 
Centres are chosen by having an intensity greater than some predetermined theshold. Once the set of centres, $C= \{c_0,c_1,...,c_n\}$, is created, it is sorted in lexical order i.e. by their $x$ values and then by the $y$ values if $x$ values are equal. Once the centres have been created, generation of the Voronoi tessellation can begin.
\\
\\
For the sake of testing the system, the $x$ and $y$ coordinates of the sources are randomly generated on a $600 \times 600$ plane with the intensities, $z$, randomly generated as the absolute value of a random normal distribution, centred at zero with a standard deviation of $3000$. Using this, $10000$ sources are randomly generated. From these sources, we generate centres to be used by the Voronoi, where centres are sources with an intensity greater than one standard deviation of the mean. Since the absolute value of the source intensity is used, this includes all sources with an intensity greater than 3000, thus accounting for approximately $32\%$ of the sources.
\begin{figure}[H]
\includegraphics[width=\textwidth]{Images/sources.png}
\caption{ All sources (green circles) with those selected as centres (red crosses).}
\label{fig:source}
\end{figure}
Figure \ref{fig:source} shows sources in green with their intensities representing size of the circle on the plane, for simplicity, only $200$ points are generated. The crosses in red represent the centres that will be used to generate the Voronoi tessellation.