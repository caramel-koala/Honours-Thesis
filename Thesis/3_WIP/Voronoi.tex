\section{Voronoi}
The current best model uses k-means clustering of of the points to the center points. For every point (the plane is seen as a finite number of pixels) in the plane, $p$, it finds the seed point, $s_i$, which is the shortest distance to it, i.e. such that $\sqrt{\big(p(x)-s_i(x)\big)^2 + \big(p(y)-s_i(y)\big)^2}$ is minimum. This is suboptimal as the time taken to create the diagram relies on the dimensions of the plane and the number of seed points in it. Instead we seek a method which is invariant of the plane size and relies solely on the seed points, for the sake of increasing the computation time, this method must also be parallelizable. 
\\
\\
It was therefore decided that the divide and conquer method (Chapter \ref{tes:ssec:dac}) would be used. The algorithm works by ordering the points, first by their $x$ then their $y$ values. The points are then divided into two subsets, a left and right set. The Voronoi diagrams are then generated for the left and right subsets using the divide and conquer method. The convex hulls of the left and right Voronoi diagrams are then found. The lowest common support line between the hulls is then found and from this a dividing polygonal chain is generated until it intercepts the upper bounds of the plane. The intersecting edges with the polygonal chain are then determined, and cut so that part of the chain is now part of the Voronoi cells edges.