This chapter will discuss the design of the tessellation algorithm. It begins by describing the improvements made by implementing the divide and conquer Voronoi algorithm as well as the details of the algorithm itself. It then discusses the error metric which will be used to test the performance of our algorithm and concludes by detailing the cell merge algorithm used to group Voronoi cells to form larger cells.
\section{Overview}
Given a large set ($\sim10000$) of extragalactic sources obtained by aperture synthesis (Section \ref{ra:sec:ic}), an optimal means to tessellate the space containing the sources into facets must be found so that the correction of DD-effects is maximal.
\\
\\
A metric is needed to calculate the effectiveness of the algorithm. Since each tessellated facet or cell contains exactly one correction centre and, potentially, multiple sources, an error metric is used. The error for each cell, $\epsilon_i$, is defined as the sum of the distance from each source to the correction centre, and the sum of all the errors per cell given the error over the entire tessellation, $E = \sum^n_i \epsilon_i$ where $n$ is the number of cells. The algorithm must therefore minimise $E$ by choosing cell boundaries and centres such that each $\epsilon_i$ is as small as possible while simultaneously keeping $n$ as low as possible.
\\
\\
A Voronoi tessellation (Section \ref{tes:sec:vor}) was used to tessellate the space into cells. However, given the large number of sources, using each source as a centre is suboptimal and defeats the purpose of finding an algorithm which minimises both the error and the number of cells needed, especially considering the large intensity differences in the sources, which can differ in the order of $10^5$ in magnitude. A smaller subset of the sources is therefore used to generate the Voronoi tessellation.
\\
\\
In cases where two large sources are close neighbours of one another, a single larger cell would be preferred over two smaller cells, this would however affect the overall error, $E$, of the tessellation. A merge algorithm was designed and used to find centres with relatively close centres leading to the lowest overall increase in the error. For the sake of the merge algorithm, it is preferable to start with the entire set of sources as centres and merge centres together until the maximum allowed error is reached or the number of cells is minimised.
\\
\\
In the following sections, each of the steps taken in implementing the sequential tesellation algorithm is explained.