\section{Contribution of Research}
Referring to the research objectives stated in Section \ref{int:sec:goals}, the contributions of the research done in this thesis are:
\begin{enumerate}
\item The divide and conquer algorithm was implemented to generate the Voronoi tessellation faster than that of the current best model which used a naive k-means clustering of all points in the space to generate the tessellation. The neighbouring structure generated by the divide and conquer was also beneficial for the cell merge algorithm implemented later. Weighting the Voronoi, however, could not be implemented
\item The error metric, $\epsilon_j = \sum^N_{i=1} z_i|\vec{x_i} - \vec{x_j}|^2$, was determined to be an appropriate means to measure the effectiveness of the tessellation algorithm.
\item The cell merge operation was determined to be a more effective means of generating tessellations for the given problem space. As shown by the results, it did tessellate more effectively than a standard Voronoi for most cases. To the best of the researchers knowledge, this is the first implementation of this algorithm for correcting direction dependant effects.
\item The GPU was used to improve the overall performance of the cell merge algorithm and showed favourable results for larger sets of sources with a speed-up of 39.96x for 1000 sources. This is also a new means in which this tessellation algorithm has been parallelised. It was however found that the parallel algorithm was limited by GPU memory and due to complications with float calculations, there were minor differences in the results obtained by the GPU.
\end{enumerate}
