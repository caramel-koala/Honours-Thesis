\section{Thesis Summary}
In this thesis an improvement to the current best model for generating tessellations for correcting direction dependent effects. We began by first researching what direction dependant effects are and why they need to be corrected for. Tessellation models, mainly the Voronoi tessellation, were researched as well as different methods of implementing Voronoi tessellations and clustering algorithms. Lastly parallel paradigms, GPUs, GPU architecture and CUDA, the NVIDIA GPU programming language were analysed for their potential contribution to this thesis.
\\
\\
The algorithm that would be used to generate the tessellation was then defined. It was decided that, optimally, the tessellation would occur in two stages. The first stage would be a Voronoi tessellation which was generated using most or all of the data sources as cell centres. For this the divide and conquer algorithm was chosen for its efficient $O(n\log n)$ computation time and also for its potential to be converted to a parallel algorithm. The second part of the tessellation used the error of the cells to find and execute optimal cell merges to lower the number of cells while increasing the overall error in a minimal way. The cell merged worked by iteratively cycling through finding the best merge to execute by testing the merge of each cell with its neighbour and, once found, executing the best merge until the maximum error threshold is reached.
\\
\\
The cell merge process within the main algorithm was ported to execute as a parallel process on the GPU. Numba's CUDA JIT compiler was used to convert specific python commands and functions to CUDA code for execution of the GPU. The conversion of the data, from a large relational network of pointers to a set of multidimensional arrays of a fixed size was discussed. The parallelisation of the merge testing algorithm was discussed as well as executing the merge on both the CPUs host memory and on the GPU. Problems with the GPU execution due to errors in float calculations were also noted.
\\
\\
Results from testing the algorithm were then analysed. It was found that for most tessellations for a given data set, the cell merge algorithm had a lower error than that of using just the Voronoi. For a very low, $\leq 3\%$, cells per source, the Voronoi then becomes more effective with lower errors. For the GPU implementation of the cell merge, it was found that the speed-up of the algorithm varied with the number of sources. For a tessellation with 10 sources, it was found that the GPU execution was slower than that of the sequential algorithm while, for 1000 sources, the speed-up was 39.96x. The cause of the change in the speed-up was found to be the overhead associated with the GPU algorithm. For a smaller set of sources, the overhead cost is relatively high but increases at a much lower rate than that of the merge execution itself, which is the main process being sped-up.