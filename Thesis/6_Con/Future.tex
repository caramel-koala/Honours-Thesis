\section{Future Work}
While the research done in this thesis did much to improve how tessellations for correcting direction dependant effects are generated, there is still much that can be done to improve on this for better results in possibly faster times.
\\
\\
The divide and conquer Voronoi tessellation algorithm could be parallelised for faster generation of the initial tessellation.
\\
\\
The effects of the intensities could be calculated in such a way that they can be incorporated into the Voronoi tessellation algorithm without leading to domain errors.
\\
\\
The error metric could be updated to include a differentiable approximation of the distortion field generated by the direction dependant gains. While more difficult and computationally expensive, this would lead to a more accurate representation of the error and the position at which it needs to be corrected.
\\
\\
The cell merge could be changed to allow for better means of dynamically assigning cells to merge. This could correct the fact that cells merges are fixed once executed and cannot be undone if a better merge becomes available at a later stage.
\\
\\
The parallel implementation, while producing positive results, was not well optimised. The algorithm could be better optimised to rely less on large arrays so that larger sets of sources can be accommodated on the GPU. The optimisations could also assist in improving the overall speed of the execution to allow for even better speed-ups.
\\
\\
Alternatively, a GPU with a larger memory could be used to handle larger sets and possibly compute the tessellation in faster time.
\\
\\
The problem surrounding the floating point operations on the GPU could also be looked into to provide a better means to generate more accurate merges.