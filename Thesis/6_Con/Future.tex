\section{Future Work}
While the research done in this thesis did much to improve how tessellations for correcting direction dependant effects are generated, there is still much which can be done to improve on this for better results in even faster time.
\\
\\
The divide and conquer Voronoi tessellation algorithm can be parallelised so that it generates a Voronoi tessellation in an even faster time.
\\
\\
The effects of the intensities can be calculated in a way such that it can still be incorporated into the Voronoi tessellation algorithm without leading to domain errors.
\\
\\
The error metric can be updated to include a differentiable approximation of the distortion field generated by the direction dependant gains. While more difficult and computationally expensive, it will lead to a more accurate representation of the error and the position at which it needs to be corrected.
\\
\\
The cell merge can be changed to allow for better means of dynamically assigning cells to merge. This could correct the cell merges are fixed once executed while a better merge might be present at a later stage.
\\
\\
The parallel implementation, while producing positive results, was not optimised very well for testing. The algorithm could be better optimised to rely less on large arrays so that larger sets of sources can be accommodated on the GPU. The optimisations could also assist in improving the overall speed of the execution to allow for even better speed-ups.
\\
\\
Alternatively, a GPU with a larger memory can be used to handle larger sets and possibly compute the tessellation in faster time.
\\
\\
The problem surrounding the float operations on the GPU can also be looked into to provide a better means to generate more accurate merges.