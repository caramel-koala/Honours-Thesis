\begin{abstract}
With the promise of the SKA comes multiple challenges in terms of capturing and cleaning the data. One part of this involves breaking up or tessellating an image so that it can be cleaned of noise for better analysis. While methods to do this are currently in circulation, more can be done to ensure the results are as accurate as possible and are obtained as quickly as possible.
\\
\\
This research seeks to improve the current best tessellation model for correcting the noise and do so in an optimal way with specialised hardware. To achieve these aims a novel algorithm is created and tested to generate the tessellation more effectively than the current best model. In order to increase the calculation speeds, part of this algorithm is then parallelised for processing on a GPU.
\\
\\
The tessellation algorithm generated for this research is more effective than the current best model in general. Through accelerating parts of the algorithm on a GPU, speed-ups of up to 39.96x were obtained for tessellations generated from 1000 data sources.
\end{abstract}

\chapter*{ACM Computing Classification System Classification}
Thesis classification under the ACM Computing Classification System\footnote{http://www.acm.org/about/class/2012/} (2012 version valid through 2016):

\begin{itemize}
\item {[500] \textbf{Theory of computation} $\rightarrow$ \em {Computational geometry} }
\item {[500] \textbf{Computing methodologies} $\rightarrow$ \em {Parallel computing methodologies} }
\item {[300] \textbf{Theory of computation} $\rightarrow$ \em {Divide and conquer} }
\item {[300] \textbf{Mathematics of computing} $\rightarrow$ \em {Combinatoric problems} }
\item {[100] \textbf{Theory of computation} $\rightarrow$ \em {Generating random combinatorial structures} }
\item {[100] \textbf{Applied computing} $\rightarrow$ \em {Astronomy} }
\end{itemize}
\textbf{General Terms:} Tessellation, Cell Merge, Voronoi, GPU, Convex Hull, Numba, Direction Dependent Effects, Radio Astronomy

\chapter*{Acknowledgements}
I would like to thank Prof. Oleg Smirnov and the Rhodes Centre for Radio Astronomy Techniques and Technologies (RATT) for funding this research and my honours degree, and to Dr. Cyril Tasse for theorising the research, its been challenging but also a lot of fun.
\\
\\
I would also like to thank my supervisors, Prof. Karen Bradshaw and Prof. Denis Pollney, without your knowledge and council, this thesis would be a mess that not even I could understand.
\\
\\
To the staff of the Hamilton Building, thank you for being the best department on campus and a special thank you to Caro Watkins for putting up with me and my ever-changing degree.
\\
\\
To my friends and family, thank you for the love and support you have given me throughout this endeavour, your words of encouragement have always driven me when I needed them the most.
\\
\\
To the Rhodes CS honours class of 2016, getting to know you all made my year, thank you for your support through all this. I wish you all the best for your futures as you go out into the world to show it what you can do.
\\
\\
Lastly, I'd like to thank my grandparents, Benjamin and Edna Draai, without your love, guidance and words of wisdom, none of this would be possible.
\\
\\
This work was undertaken in the Distributed Multimedia CoE at Rhodes University, with financial support from Telkom SA, Tellabs/CORIANT, Easttel, Bright Ideas 39, THRIP and NRF SA (UID 90243).  The authors acknowledge that opinions, findings and conclusions or recommendations expressed here are those of the author(s) and that none of the above mentioned sponsors accept liability whatsoever in this regard.