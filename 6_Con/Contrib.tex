\section{Contributions of the Research}
Referring to the research objectives stated in Section \ref{int:sec:goals}, the contributions of the research documented in this thesis are:
\begin{enumerate}
\item The divide and conquer algorithm was implemented to generate a Voronoi tessellation faster than the current best model which uses a naive k-means clustering of all points in the space to generate the tessellation. The neighbouring structure generated by the divide and conquer was also beneficial for the cell merge algorithm implemented later.
\item The error metric, $\epsilon_j = \sum^N_{i=1} z_i|\vec{x_i} - \vec{x_j}|^2$, was determined to be an appropriate means to measure the effectiveness of the tessellation algorithm.
\item The cell merge operation was shown to be a more effective means of generating tessellations for the given problem space. According to the results, it did tessellate more effectively than a standard Voronoi in most cases. To the best of the researcher's knowledge, this is the first implementation of this algorithm for correcting \gls{dd} effects.
\item The GPU was used to improve the overall performance of the cell merge algorithm and showed favourable results for larger sets of sources with a speed-up of 39.96x for 1000 sources. This presents a new way of parallelising the tessellation algorithm. It was however found that the parallel algorithm was limited by GPU memory and due to differences in floating point calculations, there were minor differences in the results obtained by the GPU. However, the tessellations generated by both the sequential and parallel algorithms are still valid given the context of the research.
\end{enumerate}
