\section{Executing the Cell Merge}
Once the best cell merge has been found and the error confirmed to be less than the maximum error threshold, the merge is executed. While the merge execution on the CPU is nearly identical to that of the sequential merge algorithm, the data changes resulting from the merge must also be reflected in the data stored on the GPU. This is necessary to ensure that the next iteration of finding the best merge on the GPU uses the updated data.

\subsection{CPU Cell Merge}
As stated the CPU merge is almost identical to that of the sequential merge except that the parallel CPU merge function takes in a list containing the positions of the cells to be merged instead of being passed pointers to their location in memory.

\subsection{GPU Cell Merge}
The best merge in the \textit{results} array is copied back to the GPU as the array \textit{d\_best} and the kernel \textit{d\_do\_merge} is called to execute the merge on the GPU with the same grid and block parameters as the \textit{d\_get\_best} kernel. The kernel begins by using the \textit{cuda.grid} function to get the global thread ID of the processing thread and checks to see if the thread is within the range of the number of cells. The \textit{d\_related} array is iterated over and any element containing the value of the related merging cell, is changed to the value of the main merging cell. For example, if cell 4 merging with cell 5 was found to be the best merge, all elements in \textit{d\_related} with a value of 5 are changed to contain 4.
\\
\\
Next, the thread ID is checked to be equal to that of the main merging cell (cell 4 in our example). This thread changes the values of the coordinates, intensity and error of the main merging cell in \textit{d\_related} to that of the merged cell from the values in \textit{d\_best}. The related cells of the related merging cell (cell 5) are appended to those of the main merging cell's list in the \textit{d\_related} array. The sources of the related merging cell are appended to that of the main merging cells in \textit{d\_sources}. Continuing the example, thread 4 changes the values in the fourth column of \textit{d\_centre} to those of the merge stored in \textit{d\_best}. The sources of cell 5 are added to those of cell 4 as cell 4 now covers the area of cell 5 as well.
\\
\\
The function finally checks the thread ID to see if it is that of the related merging cell (cell 5). This thread changes the values of the position and intensity of the cell to that of the maximum error threshold, to ensure it will be unable to merge with any other cells and sets its error to zero in the \textit{d\_centre} array. Using the example, this means that the values of the fifth column of \textit{d\_centre} are changed to be extremely large values. To make the value relative to the scale of the data, the maximum error threshold is therefore chosen, which will ensure that no other cell attempts to merge with cell 5 or if an attempt is made, it will exceed the maximum error.
\\
\\
Once these changes have been made, the kernel ends its execution and the process begins again until the maximum error is reached.