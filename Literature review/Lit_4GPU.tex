\section{GPU Architecture and Concepts}\label{gpu}
%--------------------------------------------------------------------------------------
\subsection{Parallelism}\label{gpu:sec:par}
One of the main means of optimizing processing is through parallelism. The two main forms of parallelism are task and data parallelism. Task parallelism can be seen as running multiple processes concurrently where communication between the processes is explicitly defined to avoid race conditions. Data parallelism is the distribution of a data set over a number of identical processes each of which performs operations on a unique subset of the data. Race conditions occur when parallel processing streams access data or perform operations out of the intended order, leading to errors or incorrect output being produced. A combination of task and data parallelism can lead to an ideal speed-up, but both have their limits depending on the task and the data being operated on (\cite{subhlok1993exploiting}).
\\
\\
The increased need for parallelism came about in 2005, when CPU frequency peaked at 4 GHz due to heat dissipation issues. However, Moore's Law still holds, and is still expected to hold until 2025; that is, that the number of transistors for a computer will double every two years. This leads to a problem where the speed at which an operation is done cannot be increased (due to the frequency limit), but the number of concurrent operations can still increase. This means that the only way to speed up an operation is to change it from a sequential to a parallel process (\cite{rajan2013informatics}).
%--------------------------------------------------------------------------------------
\subsection{GPU Optimization}\label{gpu:sec:opt}
Graphical Processing Units (GPUs) were originally designed for rendering pixels and vectors in games. They were especially designed for this since CPU's are optimized to run sequential instructions as fast as possible; whereas pixel and vector calculations are inherently parallel. With NVIDIA's release of CUDA in 2006, general purpose GPU (GPGPU) programming became common place as a way to accelerate data processing through data parallelism and task parallelism through the simultaneous execution of similar tasks (\cite{nvidia_cuda}).
\\
\\
The power of GPUs comes from its architecture which is optimized for a special case of SIMD (Single Instruction Multiple Data) processing known as SIMT (Single Instruction Multiple Threads). SIMD allows a central processor to distribute a set of instructions to multiple simple processors which then act on the data simultaneously. SIMT is more generalized as each warp of the GPU can perform different tasks given the same set of instructions. This is due to the way in which the GPU handles branching at the thread level. By exploiting these processes, and this instructional architecture, some instructions can be computed faster than a CPU (\cite{vuduc2013brief}).
%--------------------------------------------------------------------------------------
\subsection{The NVIDIA GeForce GTX 750 Ti}\label{gpu:sec:750}
%
\subsubsection{GM107 Maxwell Architecture}\label{gpu:ssec:max}
The NVIDIA GeForce GTX 750 Ti GPU was released on the 18th of February 2014. It boasts 640 CUDA cores, 1020 MHz base clock speed, 1305.6 GFLOPs and a memory bandwidth of 86.4 GB/sec. It is NVIDIA's first-generation Maxwell architecture, designed for high performance at relatively low power consumption (60 W) and has the codename 'GM107'. The GPU uses PCI Express 3.0 to interface with the host machine through the GigaThread engine. The first-generation Maxwell (from now simply referred to as Maxwell) is made up of one Graphics Processing Cluster (GPC) on which the processing occurs. It also contains a large L2 cache at 2048 KB and two 64-bit memory controllers to access the 2048 MB global memory. This design can be seen in Figure \ref{gpu:img:gm107} (\cite{geforce_750}, \cite{g750_specs}, \cite{g750_paper}).
%
\subsubsection{Streaming Multiprocessors}\label{gpu:ssec:smm}
The GPC is further broken down into five streaming multiprocessors(SMs) which are further divided into four processing blocks. The processing blocks (or warps) each contain an instruction buffer, a scheduler and 32 CUDA cores as seen in Figure \ref{gpu:img:smm} (\cite{GM107SMM}). These warps are set in a lock step, meaning each core in a warp executes the same set of commands at the same time, with different valued variable.(\cite{CUDA_DEVKIT}
%
\begin{figure}[H]
 \centering
 %\includegraphics[width=0.7\textwidth]{Images/GM107.jpg}
 \caption{NVIDIA Maxwell Architecture(\cite{GM107})}
 \label{gpu:img:gm107}
\end{figure}
%
\begin{figure}[H]
\centering
 %\includegraphics[width=0.5\textwidth]{Images/GM107SMM.png}
 \caption{Maxwell Streaming Multiprocessor(\cite{GM107SMM})}
 \label{gpu:img:smm}
\end{figure}
%--------------------------------------------------------------------------------------
\subsection{CUDA}\label{soft:sec:cuda}
CUDA is a parallel programming language created by NVIDIA for the purpose of running on their brand of GPU's. CUDA was modeled as a C-like language with some C++ features. Its main feature is the way in which it separates CPU and GPU code. The CPU code is labeled as ``host'' code and the GPU's as ``device'' code. Device code is called by the host through a special case of a method, known as a kernel. The basic structure of a kernel is as follows: 
\\
\\
\texttt{kernel0<<<grid, block>>>(params);}
\\
\\
In this instance \texttt{kernel0} would be the name of the kernel being called, \texttt{grid} is the three dimensional value of the number of blocks to be assigned, \texttt{block} being similar to \texttt{grid} is a three dimensional value of the number of threads needed and \texttt{params} is simply the parameters needed by the kernel to execute (similar to those of a method) (\cite{CUDA_DEVKIT}).
%
\subsubsection{Threads}\label{soft:ssec:thread}
The thread is the smallest processing unit of the GPU. GPU threads are designed to be cheap and lightweight compared to those of a CPU so that it can be easily created, run it's small task and be destroyed to make place for the next thread. Threads are arranged into three dimensional blocks with each thread having a unique 3 dimensional ID within that block, namely an x, y and z ID. Generally the thread ID is used as the means of determining the difference in the task process of each thread (\cite{CUDA_DEVKIT}).
%
\subsubsection{Blocks}\label{soft:ssec:block}
Each block may have a maximum of 2056 threads in total and 1024 for any single dimension, hence why they are bundled into a larger, three dimensional grid structure. Similarly to threads, blocks have a unique three dimensional ID in the grid. Blocks exist such that each step of the processes execute simultaneously. This is done as, more often than not, blocks exchange data within their threads and if this precaution is not taken, race conditions could ensue to break the code. Each block, when executing, must occupy a whole number of warps (rounded up). This is done as warps are constantly in lock step. Threads within a block share a fast memory, located in the L1 cache of the streaming multiprocessor. This shared memory must be preallocated when the kernel is called as a third parameter within the kernel launch (parameters within the triple angle brackets) (\cite{CUDA_DEVKIT}).
%
\subsubsection{GPU Memory Hierarchy}
In order to maximize concurrency on the card, the GPU has a structured memory hierarchy. The largest, slowest and most generally accessible of these is the global memory which lies in the device memory. This memory is visible to every thread and also to every kernel called in one application. 
\\
The constant memory also lies on the device memory, as the name suggests, values stored here cannot be altered and are read-only until the space is deallocated. Variables stored in constant memory also have the ability to broadcast their values to multiple threads simultaneously.
\\
Similar to constant memory, texture memory also lies on the device memory and is also read-only, it is designed for storing arrays where multiple neighboring values of the array can be read concurrently.
\\
The shared memory lies on the SM and is visible only to a block as a means for threads within a block to exchange data. Shared memory must be declared with the size of the memory needed (up to the maximum $64$Kb) when the kernel executed.
\\
Each processor is assigned its own memory to be used by each thread, these are called registers. Registers are visible only by the thread currently on that processor and reset with each change of thread. Registers hold the variables created in and passed to the thread. The register is by far the smallest memory on the card at 32 bits per register, but 255 registers per thread. Should the thread call too many variables or variables too large to fit in the registers, then the variables spill over into local memory. local memory lies in the L1 and L2 cache for active threads. Should the thread need to be temporarily halted for another thread to use the processor, then the threads variables are stored in local memory on the device memory \citep{CUDA_DEVKIT}.
%
\subsubsection{CUDA Optimization}\label{gpu:ssec:cop}
%--------------------------------------------------------------------------------------
\subsection{Related Works}\label{gpu:sec:rel}
